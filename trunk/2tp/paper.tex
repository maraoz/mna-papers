%
% tp 2 mna
% 

% Use emulateapj instead to make Apj format.
% YOU SHOULD REMOVE THE TABLE OF
% CONTENTS PAGE WHEN USING APJ FORMAT.
\documentclass[11pt,a4paper]{emulateapj}
\bibliographystyle{apj}


%define general packages
\usepackage{epsfig}
\usepackage{amsmath}
\usepackage{natbib}

% spanish packages
\usepackage[utf8]{inputenc}
\usepackage[spanish]{babel}
\languageshorthands{none}
\noextrasspanish
\let\layoutspanish\relax
\usepackage[spanish]{babel}
\renewcommand\shorthandsspanish{}

%internal short cuts
\def \HgA {H$\gamma_A$}
\def \gon {Gonz\'{a}lez}
\def \Hbp {H$\beta ^\prime$}
\def \warn {{\sffamily\bfseries\large WARNING, ARREGLAR:}}









\begin{document}

\submitted{Departamento Ing. en Informática, ITBA}
\title{Filtros espaciales en imagenes mediante el paso del dominio de los colores al dominio de las frecuencias}
\author{Williams M. \& Aráoz M.}
\date{\today}


\begin{abstract}
\warn{hacer abstract}
\end{abstract}

\maketitle




\section{Introduccción}
\label{sec:introduccion}
Un reactor nuclear de fisión es un dispositivo en el que se producen reacciones nucleares
donde neutrones rompen núcleos de número atómico grande. En dicha ruptura, se
producen: 
\begin{enumerate}
	\item fragmentos de fisión, esto es núcleos con Z intermedio.
	\item neutrones que se utilizan para continuar el proceso de fusión y 
	\item energía del orden de 200MeV. Si se dan
las condiciones necesarias, este proceso es autosostenido, produciéndose una reacción en cadena.
\end{enumerate}
La energía liberada permite calentar agua que circula por un determinado circuito. Así
se genera vapor. Este vapor es utilizado para producir el movimiento de turbinas en
generadores de electricidad.

La difusión de neutrones en el núcleo de un reactor debe ser controlada, para poder mantener
la reacción en cadena. Para tal fin se utilizan medios moderadores como el agua $H_2O$
o el agua pesada $D_2O$.


En reactores autoregenerables se utiliza como combustible nuclear sales de $U^{238}_{92}$ .

Para modelar el proceso estacionario de difusión de neutrones en un reactor, consideremos
un modelo de celda de combustible y medio moderador unidimensional. La sal $UO_2$ se
encuentra en el núcleo de la celda emitiendo neutrones. Tales neutrones pasan al medio
moderador que controla la difusión. Consideremos a dicho medio agua liviana.

La ecuación de difusión nuclear para los neutrones, en estado estacionario es:
\begin{eqnarray}
\label{eq:ecuacionADiscretizar}
  -\dfrac{d}{dx}[D \dfrac{d\phi}{dx}(x)] + \sigma \phi(x)  &=& \lambda  \Sigma_f \phi(x)
\end{eqnarray}
donde $\phi$ es la concentración de neutrones, $D$ es el coeficiente de difusión neutrónica del
medio, $\sigma$ es el coeficiente de absorcion de neutrones y $\Sigma_f$ es la sección eficaz de fusión del
medio. La constante $\lambda$ depende de la geometría del reactor y por lo tanto es un parámetro
de diseño.
Para el modelo unidimensional compuesto por una celda junto a un moderador 
-cuyo diagrama se muestra en la figura \ref{fig:esquema}- las dimensiones 
lineales son $20cm$ para la celda de combustible y 10cm para la barra de moderador.
Además, en la figura, se dán los valores de $\sigma$, $D$, y $\Sigma_f$. \citet{diaz10}

%\begin{figure*}
%  \begin{center}
%    \leavevmode
%      \psfig{file=images/img1.png, width=254px}
%       \caption[Esquema simple del reactor]{Modelo de un ractor nuclear unidimensional.
%         Diagrama adaptado de \citet{diaz10}.}
%     \label{fig:esquema}
%  \end{center}
%\end{figure*}

Como se puede apreciar, los parámetros que se mantienen constantes dentro de cada compartimiento, pueden verse
como una función constante por tramos que varía según $x$. Tenemos entonces:

\begin{eqnarray}
	D(x)=\left\{
		\begin{matrix}
			2 &\mbox{ si }& 0 < x < 20\\
			1 & \mbox{ si }& 20 < x < 30\\
		\end{matrix} \right.
\end{eqnarray}
\begin{eqnarray}
	\sigma(x)=\left\{
		\begin{matrix}
			0.1 &\mbox{ si }& 0 < x < 20\\
			0.021 & \mbox{ si }& 20 < x < 30\\
		\end{matrix} \right.
\end{eqnarray}
\begin{eqnarray}
	\Sigma_f(x)=\left\{
		\begin{matrix}
			0.11 &\mbox{ si }& 0 < x < 20\\
			0 & \mbox{ si }& 20 < x < 30\\
		\end{matrix} \right.
\end{eqnarray}

Por ende se puede reescribir la ecuación diferencial \ref{eq:ecuacionADiscretizar} de la siguiente manera:
\begin{eqnarray}
\label{eq:ecuacionADiscretizarParticular}
  -\dfrac{d}{dx}[D(x) \dfrac{d\phi}{dx}(x)] + \sigma(x) \phi(x)  &=& \lambda  \Sigma_f(x) \phi(x)
\end{eqnarray}
En este trabajo se halla una solución numérica para la ecuación diferencial anterior. El vector solución
representa la concentración de neutrones $\phi(x)$.

El resto de este trabajo se organiza de la siguiente manera: en la sección \ref{sec:discretizacion} se discretiza
el problema continuo mediante un algoritmo para el cómputo de la derivada y en la sección \ref{sec:computo} 
se utiliza este modelo discretizado para hallar una solución numérica para la concentración de neutrones
en estado estacionario del reactor nuclear. En la sección \ref{sec:resultadosyconclusiones} se presentan las conclusiones
que se desprendieron de este trabajo.

\section{Discretización del problema}
\label{sec:discretizacion}
Vamos a discretizar el problema continuo que se presenta en la ecuación (\ref{eq:ecuacionADiscretizar}) 
utilizando un esquema centrado de orden dos. Para lograr esto se utilizará el algoritmo (\ref{eq:algoritmoDeDiscretizacion}),
\begin{eqnarray}
\label{eq:algoritmoDeDiscretizacion}
	f'(x) = \frac{f(x+h) - f(x-h)}{2h} - \frac{1}{12} f'''(\xi) h^2
\end{eqnarray}
donde $\xi \in int\{x-h,x+h\}$ y $h$ es el tamaño de un intervalo, el parámetro computacional.
Ahora entonces procedemos a discretizar el problema. Lo resolveremos por partes para simplificar los cálculos. Primero buscamos una aproximación para la expresión \ref{eq:discretizacionDelaDerivada}.
\begin{eqnarray}
\label{eq:discretizacionDelaDerivada}
	\dfrac{d}{dx}\bigg[D \dfrac{d\phi}{dx}(x)\bigg]  
\end{eqnarray}
Como ya dijimos, el valor de $D$ depende del valor de $x$, y la denotaremos $D(x)$. La misma expresión \ref{eq:discretizacionDelaDerivada}, se puede escribir, entonces, de la siguiente forma:
\begin{eqnarray}
\label{eq:discretizacionDelaDerivadaMejorado}
	\dfrac{d}{dx}[D(x) \phi'(x)]
\end{eqnarray}
Si aplicamos el algoritmo \ref{eq:algoritmoDeDiscretizacion} en la ecuación \ref{eq:discretizacionDelaDerivadaMejorado}, se obtiene la siguiente expresión:
\begin{eqnarray}
\label{eq:desarrolloDeLaDiscretizacion}
	\dfrac{d}{dx}[D(x) \phi'(x)] &=& \dfrac{D_{i+\frac{1}{2}}\phi'\big(x_{i+\frac{1}{2}}\big)  -  D_{i-\frac{1}{2}}\phi'\big(x_{i-\frac{1}{2}}\big)}     {2h} 
\end{eqnarray}
Donde hemos considerado:

\begin{eqnarray}
	x_i &=&x_0 + h*i \\
	\phi_i &\cong&\phi(x_0 + h.i)
\end{eqnarray}

De la misma manera, si aplicamos el algoritmo \ref{eq:algoritmoDeDiscretizacion} sobre los valores $\phi'(x_{i+\frac{1}{2}})$ y $\phi'(x_{i-\frac{1}{2}})$ en la ecuación \ref{eq:desarrolloDeLaDiscretizacion}, obtenemos el siguiente resultado:
\begin{eqnarray}
\label{eq:desarrolloDeLaDiscretizacion2}
	\dfrac{1}{2h} \bigg[	D_{i+\frac{1}{2}}.\bigg(\dfrac{\phi_{i+1} - \phi_{i}}{2h}\bigg) 	-
			D_{i-\frac{1}{2}}.\bigg(\dfrac{\phi_{i} - \phi_{i-1}}{2h}\bigg) \bigg]\\
\end{eqnarray}
Que se puede reordenar como:
\begin{eqnarray}
\label{eq:reemplazoDeLaDerivada}
	\dfrac{D_{i+\frac{1}{2}}(\phi_{i+1} - \phi_{i}) - D_{i-\frac{1}{2}}(\phi_{i} - \phi_{i-1})}{4h^2}
\end{eqnarray}
Si escribimos de nuevo la ecuación, pero ahora teniendo en cuenta la discretización recién realizada logramos la siguiente expresión:
\begin{eqnarray}
\label{eq:discretizacionNoFinal}
	-\bigg[\dfrac{D_{i+\frac{1}{2}}(\phi_{i+1} - \phi_{i}) - D_{i-\frac{1}{2}}(\phi_{i} - \phi_{i-1})}{4h^2}\bigg]
	+ \sigma_i \phi_i &=& \lambda \Sigma_{fi} \phi_i
\end{eqnarray}

Si realizamos cuentas sobre la ecuación \ref{eq:discretizacionNoFinal}, obtenemos la forma final discretizada:
\begin{eqnarray}
	-D_{i+\frac{1}{2}}.\phi_{i+1} + \\
	+(D_{i+\frac{1}{2}} + D_{i-\frac{1}{2}} + 4h^2.\sigma_i)\phi_{i} - \\
	-D_{i-\frac{1}{2}}\phi_{i-1} &=& \lambda 4h^2 \Sigma_{fi} \phi_i
\end{eqnarray}
Las condiciones de borde dadas son $\phi'(0) = 0$ y $\phi(30) = 0$. Asimismo, hay que tener en cuenta que para el caso $i = 0$ podemos aproximar la derivada como:
\begin{eqnarray}
	\phi'(0) &\cong & \dfrac{\phi_1 - \phi_0}{2h}
\end{eqnarray} \\
De lo que se deduce que
\begin{eqnarray}
	\phi_0 &=& \phi_1
\end{eqnarray}
Entonces, utilizando un esquema centrado de orden 2, se logró discretizar el problema continuo que se presentó en la sección \ref{sec:introduccion}. La discretización obtenida es la siguiente:
\begin{eqnarray}
\label{eq:expresionDicretizacion}
	-D_{i+\frac{1}{2}}.\phi_{i+1} + \\
	+(D_{i+\frac{1}{2}} + D_{i-\frac{1}{2}} + 4h^2.\sigma_i)\phi_{i} - \\
	-D_{i-\frac{1}{2}}\phi_{i-1} &=& \lambda 4h^2 \Sigma_{fi} \phi_i
\end{eqnarray}
, con $\phi_0 = \phi_1$ y $\phi(30) = 0$.




\section{Cómputo de la concentración de neutrones en estado estacionario}
\label{sec:computo}
Una vez discretizado el problema, podemos computar la concentración de neutrones en estado estacionario. Esto es el vector solución del sistema de ecuaciones que se generan a partir de la expresión \ref{eq:expresionDicretizacion}, tomando los valores $i = 1,2,\ldots,29$. Esto quiere decir que el $h$ elegido es $h=1$, es decir, vamos a obtener 30 puntos, los cuales son una aproximación a la concentración de neutrones en estado estacionario para los valores $x=0cm,1,cm\ldots,30cm$. Si escribimos el sistema de ecuaciones en forma matricial obtenemos el siguiente sistema:
\begin{eqnarray}
\label{eq:sistemasdeecuaciones}
A \cdot \phi = 4h^2 \lambda \Sigma_f \phi
\end{eqnarray}
Donde A es la siguiente matriz:
\begin{eqnarray}
\label{eqn:matrizA}
	 \left( \begin{array}{cccc}
		D_{i+\frac{1}{2}} + 4h^2\sigma_i & -D_{i+\frac{1}{2}}  					& 0 		&  \dots \\
		-D_{i-\frac{1}{2}} 		& D_{i+\frac{1}{2}} + D_{i-\frac{1}{2}} + 4h^2\sigma_i 	& -D_{i+\frac{1}{2}} &  \dots \\
		0 & -D_{i-\frac{1}{2}}& \ddots & \ddots \\
		\vdots &\vdots&\ddots&\ddots\\
		\end{array} 
	\right)
\end{eqnarray}
$\phi$ es el siguiente vector:
\begin{eqnarray}
\phi = (\phi_1 \quad \phi_2 \quad \phi_3 \quad \phi_4 \quad \cdots\quad \phi_{29} )^T
\end{eqnarray}
Y $\Sigma_f$ es la siguiente matriz:
\begin{eqnarray}
\label{eqn:matrizA}
	 \left( \begin{array}{ccccc}
		\Sigma_{f1} & 0 & 0 & \cdots & 0\\
		0 & \Sigma_{f2} & 0 &\cdots & 0\\
		0 & 0 & \Sigma_{f3} &\cdots & 0\\
		\vdots & & & \ddots & \vdots \\
		0 & 0 & 0 & \cdots & \Sigma_{f29}
		\end{array} 
	\right)
\end{eqnarray}
Si resolvemos el sistema de ecuaciones \ref{eq:sistemasdeecuaciones} aplicando el método de la potencia para cálculo de autovectores y autovalores, el autovector obtenido es el vector $\phi$, que representa la concentración de neutrones en estado estacionario. \\
Sin embargo, no podemos aplicar el método de la potencia directamente, ya que el sistema de la ecuación \ref{eq:sistemasdeecuaciones} no es de la forma de un problema típico de autovalores. Esto es, al no ser de la siguiente forma:

\begin{eqnarray}
\label{eqn:problemaAutovalores}
A\phi &=& \lambda \phi
\end{eqnarray}
se debe aplicar un método modificado. El algoritmo de la potencia modificado que usamos para este trabajo se puede simplificar en:
\begin{eqnarray}
\label{eqn:algoritmoPotenciaModificado}
x_i^{(k)} &=& \Sigma_{fi}.\phi_i^{(k)} \\
y^{(k+1)} &=& Ax^{(k)}  \\
\phi^{(k)} &=& \dfrac{y^{(k)}}{||y^{(k)}||} \text{ si } k \neq 0
\end{eqnarray}

Donde $\phi^{(0)}$ es un input para el algoritmo y debe ser provisto por el usuario como una primera aproximación al autovalor. En nuestro caso se usó
el vector $\phi^{(0)} = (1\quad1\quad1\quad1\quad\dots\quad1)^T$. Con este algoritmo modificado, el valor de $\phi^{(k)}$ tiende a el valor real de $\phi$
cuando $k\rightarrow\infty$. La implementación que usamos nosotros en particular terminaba el algoritmo numérico cuando el error era menor que un cierto
parámetro de entrada. 
\section{Resultados y Conclusiones}
\label{sec:resultadosyconclusiones}

Luego de resolver por el método de la potencia modificado el sistema de ecuaciones \ref{eq:sistemasdeecuaciones} se computó una aproximación a la concentracion de neutrones en estado estacionario. Los valores exactos para cada $x$ (teniendo en cuenta que $x = k$ para el $h$ elegido) pueden apreciarse en la tabla \ref{tabla:resultados}.
\begin{table}
\begin{center}
\begin{tabular}{c|c}
\label{tabla:resultados}
 k 	&	$\phi_k$ \\
	\hline
 0 &  0.0799\\
 1 &    -0.2376\\
 2 &     0.3893\\
 3 &    -0.5311\\
 4 & 0.6593\\
 5 &    -0.7708\\
 6 &     0.8626\\
 7 &    -0.9326\\
 8 &     0.9787\\
 9 &    -1.0000\\
 10 &     0.9958\\
 11 &    -0.9663\\
 12 &     0.9122\\
 13 &  -0.8349\\
 14    & 0.7363\\
 15   & -0.6190\\
  16  &  0.4859\\
 17 &   -0.3405\\
 18 &    0.1864\\
 19 &   -0.0551\\
 20 &    0.0033\\
  21&         0\\
  22&         0\\
  23&         0\\
  24&         0\\
  25&         0\\
  26&         0\\
  27&         0\\
  28&         0\\
  29&         0\\
\end{tabular}
\end{center}
\caption{Vector $\phi$ resultante de aplicar el algoritmo de la potencia modificado. La primer columna indica el 
valor de $k$, que se corresponde con el de $x$ para el $h$ elegido, y la segunda columna el valor aproximado de 
$\phi(x)$ obtenido en ese punto.}
\end{table}


Si graficamos $\phi$ en función de la distancia ($x$), obtenemos el gráfico \ref{fig:graficoNeutrones}. Los resultados obtenidos no son los deseados pues no representan la realidad física del problema, ya que la concentración de neutrones por definición no puede ser negativa, como se muestra en los resultados obtenidos. Esto puede ser por el valor que tomamos para $h$, pero para descartar esta suposición resolvimos el sistemas de ecuaciones \ref{eq:sistemasdeecuaciones} para distintos valores de $h$, por ejemplo para $h=0.5$ y $h=0.1$. A pesar de variar los valores de $h$, no logramos obtener una aproximación a la concentración de neutrones en estado estacionario donde todos sus valores fueran positivos.
Otro factor que pudo haber llevado a esta aproximación sin correlato físico fue la tolerancia al error que se tomó para el calculo $(10^{-5})$, o que el algoritmo modificado que utilizamos no fue probado en rigurosidad para saber si era correcto para hayar la solución al sistema. \\

Trabajos futuros continuando nuestra línea podrían investigar la obtención de nuevos métodos para la resolución de este problema y quizá hallar un valor de $h$ que resulte en valores razonables para el vector solución.

%
% References
%
\bibliography{paper}

\end{document}

