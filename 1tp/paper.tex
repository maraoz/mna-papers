%
% Latex example file for postgrads (04/10/03)
%
% If you have questions please email me: 
% M.L.Balogh@durham.ac.uk
% or find me in room OC312.
%
% Note you can get Latex style files etc. from http://www.ctan.org


% Use emulateapj instead to make Apj format.
% YOU SHOULD REMOVE THE TABLE OF
% CONTENTS PAGE WHEN USING APJ FORMAT.
\documentclass[11pt,a4paper]{emulateapj}
\bibliographystyle{apj}


%define general packages
\usepackage{epsfig}
\usepackage{amsmath}
\usepackage{natbib}

% spanish packages
\usepackage[utf8]{inputenc}
\usepackage[spanish]{babel}
\languageshorthands{none}
\noextrasspanish
\let\layoutspanish\relax
\usepackage[spanish]{babel}
\renewcommand\shorthandsspanish{}

%internal short cuts
\def \HgA {H$\gamma_A$}
\def \gon {Gonz\'{a}lez}
\def \Hbp {H$\beta ^\prime$}
\def \warn {{\sffamily\bfseries\large WARNING, ARREGLAR:}}









\begin{document}

\submitted{Departamento Ing. en Informática, ITBA}
\title{Diseño de reactores nucleares \HgA}
\author{Williams M. \& Aráoz M.}
\date{\today}


\begin{abstract}
Entre 2 y tres oraciones.
\warn completar
ABSTRACT ABSTRACT ABSTRACT ABSTRACT ABSTRACT ABSTRACT ABSTRACT ABSTRACT ABSTRACT 
ABSTRACT ABSTRACT ABSTRACT ABSTRACT ABSTRACT ABSTRACT ABSTRACT ABSTRACT ABSTRACT 
\end{abstract}

\maketitle




\section{Introduccción}
Un reactor nuclear de fisión es un dispositivo en el que se producen reacciones nucleares
donde neutrones rompen núcleos de número atómico grande. En dicha ruptura, se
producen: 
\begin{enumerate}
	\item fragmentos de fisión, esto es núcleos con Z intermedio.
	\item neutrones que se utilizan para continuar el proceso de fusión y 
	\item energía del orden de 200MeV. Si se dan
las condiciones necesarias, este proceso es autosostenido, produciéndose una reacción en cadena.
\end{enumerate}
La energía liberada permite calentar agua que circula por un determinado circuito. Así
se genera vapor. Este vapor es utilizado para producir el movimiento de turbinas en
generadores de electricidad.

La difusión de neutrones en el núcleo de un reactor debe ser controlada, para poder mantener
 la reacción en cadena. Para tal fin se utilizan medios moderadores como el agua $H_2O$
o el agua pesada $D_2O$.


En reactores autoregenerables se utiliza como combustible nuclear sales de $U^{238}_{92}$ .

Para modelar el proceso estacionario de difusión de neutrones en un reactor, consideremos
un modelo de celda de combustible y medio moderador unidimensional. La sal $UO_2$ se
encuentra en el núcleo de la celda emitiendo neutrones. Tales neutrones pasan al medio
moderador que controla la difusión. Consideremos a dicho medio agua liviana.

La ecuación de difusión nuclear para los neutrones, en estado estacionario es:
\begin{eqnarray}
  -\dfrac{d}{dx}[D \dfrac{d\phi}{dx}(x)] + \sigma \phi(x)  &=& \lambda  \Sigma_f \phi(x)
\end{eqnarray}
donde $\phi$ es la concentración de neutrones, $D$ es el coeficiente de difusión neutrónica del
medio, $\sigma$ es el coeficiente de absorcion de neutrones y $\Sigma_f$ es la sección eficaz de fusión del
medio. La constante $\lambda$ depende de la geometría del reactor y por lo tanto es un parámetro
de diseño.
Para el modelo $1D$ compuesto por una celda junto a un moderador 
-cuyo diagrama se muestra en la figura N(\warn poner referencia)- las dimensiones 
lineales son $20cm$ para la celda de combustible y 10cm para la barra de moderador.
Además, en la figura, se dán los valores de $\sigma$, $D$, y $\Sigma_f$.

\section{Sección 2}

This is an example of a \LaTeX\ document.  

\medskip

Galaxies are one of the \textbf{grandest} structures of the universe.
Their sizes range from as small as $10^6 \text{M}_\odot$ to as much as
$10^{13} \text{M}_\odot$, being built of stars, interstellar gas, dust
and perhaps dark matter. Yet galaxies were recognized only early in
this century (1920s) as \textit{vast} assemblages of stars, distinct
from our own Galaxy. Edwin~P.~Hubble (1889~-~1953) was one of the
pioneers in this field. He introduced a classification diagram for
galaxies, now widely known as the Hubble ``tuning fork'' diagram
\cite[][see Figure~\ref{fig:hubble_dia} for a revised Hubble
diagram]{hub36}.


%
% Subsection heading
%
\subsection{Example of a list and how to run \LaTeX} 

% list environment
\begin{enumerate}
  
\item Write document: emacs test.tex $\Longrightarrow$ test.tex
  
\item \LaTeX\ document: latex test.tex $\Longrightarrow$ test.dvi

\item Create bibliography: bibtex test.tex $\Longrightarrow$ test.bbl

\item Correct X-refs: redo step 2. twice more!
  
\item View document: xdvi test.dvi 

\item Produce ps file: dvips test.dvi $\Longrightarrow$ test.ps

\item Print document: lpr test.ps

\end{enumerate}

%
% Another subsection
%
\subsection{Math stuff}

For the following discussion let $G(n)$ be the spectrum of a galaxy
whose redshift and velocity dispersion are to be determined and $S(n)$
the spectrum of a {template} star at zero redshift, similar spectral type
to the galaxy and at instrumental resolution. The spectra have been
continuum subtracted and end-masked, and are each sampled in N bins of
equal $\log \lambda$ spacing. A velocity shift is then a linear
function of the spacing $\Delta \log \lambda$, with velocity increments
$\Delta v$ as follows:

%
% ``Pure'' equations
%
\begin{eqnarray}
  \Delta v &=& c \times \Delta \ln \lambda \\
           &=& c \times \ln (10) \times \Delta \log \lambda
\end{eqnarray}
where c is the speed of light. Now the normalized cross correlation
function is defined as:

\begin{equation}
  \label{equ:cross_corr}
  c(n)=G(m) \otimes S(n) = \frac{1}{N\sigma_G\sigma_S}\sum_{m=0}^{N}G(m)S(m-n)
\end{equation}
where $\sigma_G$ and $\sigma_S$ are the RMS deviations:

\begin{eqnarray}
  \sigma_G^2 &=& \frac{1}{N}\sum_{n=0}^{N}G(n)^2 \\
  \sigma_S^2 &=& \frac{1}{N}\sum_{n=0}^{N}S(n)^2 
\end{eqnarray}

%
%
\subsection{Table environment}
%
%

%
% Table environment
%
\begin{table}
  \caption{A Collection of Distance Measurements to Fornax}
  \label{tab:distance}
  \begin{center}
    \leavevmode
    \begin{tabular}{lll} \hline \hline              
  $m-M$          & Method              & Reference      \\ \hline 
  31.31$\pm$0.27 & Cepheid             & \citet{sil98}  \\
  31.22$\pm$0.06 & I-band SBF          & \citet{jen98}  \\
  31.32$\pm$0.24 & K$^\prime$-band SBF & \citet{jen98}  \\
  30.94$\pm$0.33 & I-band TFR          & \citet{bure96} \\
  31.07$\pm$0.13 & V-band GCLF         & \citet{Koh96}  \\
  30.85$\pm$0.11 & I-band GCLF         & \citet{Koh96}  \\
  31.14$\pm$0.14 & PNLF                & \citet{mcm93}  \\ \hline
  \multicolumn{3}{l}{}                                             \\       
  \multicolumn{3}{l}{SBF: Surface Brightness Fluctuations}         \\
  \multicolumn{3}{l}{TFR: Tully-Fisher relation}                   \\
  \multicolumn{3}{l}{PNLF: Planetary Nebulae Luminosity Functions} \\
  \multicolumn{3}{l}{GCLF: Globular Cluster Luminosity Functions}  \\
    \end{tabular}
  \end{center}
\end{table}

%
%
\subsection{Fonts and character sizes}
%
%
One can typeset a text {\sffamily\bfseries\large in a large sans serif
  bold typeface.} Or one can use a {\slshape\tiny tiny slanted
  typeface.} For more information see ``The \LaTeX\ Companion'' page
170 .


%
%
\subsection{\LaTeX\, info}\label{sec-webrefs}
%
%
\subsubsection{WEB info on \LaTeX}
There are links to some useful websites and \LaTeX\ resources are available
from: \\ {\tt http://star-www.dur.ac.uk/$\sim$balogh/latex/}.

\subsubsection{Books and guides for \LaTeX}
\begin{itemize}
\item {\it The LaTeX Companion}, by Michel Gossens, Frank Mittelbach, Alexander Samarin.
\item {\it LaTeX: A Document Preparation System -- User's Guide and
    Reference Manual}, by Leslie Lamport, 2nd ed
\item {\it The Not So Short Introduction to LaTeX2e}, Tobias Oetiker,
  see Starlink WEB page
\end{itemize}
\subsection{Bibilographies}

Bibliographies can be greatly facilitated with \LaTeX.  The best option is to keep a database
of references: see the file http://star-www.dur.ac.uk/$\sim$balogh/latex/example.bib for an example.
Use the apropriate bibliography style file (*.bst) to format the bibliography; read the first few
lines of this file for a description of usage.  Running ``bibtex'' on your document (see below)
will generate a *.bbl file with the appropriate references.  Alternatively, you can enter the
references directly, as described in the ``Not-so-short Guide''.  

Here are some examples of the different ways you can reference a source with the
{\it natbib.sty} style file.  See the header of that file for more options.
\begin{itemize}
\item A first attempt to model this was made by \citet{bar96}.
\item This did not agree with earlier models \citep{bar96}.
\item There are several models for this phenomenon \citep{bar96,bau96}
\item There are several models for this phenomenon \citep[e.g., ][]{bar96,bau96,bure96}
\item Our model is better \citep[Fig. 5]{bau96}.
\item I want to do the references myself! (Baugh et al. 1996)\nocite{bau96}.
\end{itemize}


\section{Conclusiones}
Acá poner conclusiones, OBLIGATORIAMENTE
%
% References
%
\bibliography{paper}

\end{document}

